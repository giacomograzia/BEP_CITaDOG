\documentclass[11pt]{article}

\usepackage[sort]{natbib}
\usepackage{fancyhdr}

% you may include other packages here (next line)
\usepackage{enumitem}
\usepackage{hyperref}



%----- you must not change this -----------------
\oddsidemargin 0.2cm
\topmargin -1.0cm
\textheight 22.0cm % Reduced height to create more space at the bottom
\textwidth 15.25cm
\footskip 1.5cm % Increase this value for more space between text and footer
\parindent=0pt
\parskip 1ex
\renewcommand{\baselinestretch}{1.1}
\pagestyle{fancy}
%---------------------




% enter your details here----------------------------------

\lhead{\normalsize \textrm{BEP Draft Project Plan}}
\chead{}
\rhead{\normalsize TU/e SNR 1917811 }
\lfoot{\normalsize \textrm{JBP000}}
\cfoot{}
\rfoot{Giacomo Grazia}
\setlength{\fboxrule}{4pt}\setlength{\fboxsep}{2ex}
\renewcommand{\headrulewidth}{0.4pt}
\renewcommand{\footrulewidth}{0.4pt}

	
\begin{document}


%----------------your title below -----------------------------

\begin{center}
{\bf BEP Final Project Plan}
\end{center}


%---------------- start of document body------------------
\textbf{Project Planning}

\begin{table}[ht]
\centering
\begin{tabular}{|c|p{12cm}|}  % Adjust column width to fit the page
\hline

\textbf{Week} & \textbf{Task Description} \\ \hline
1,2 & Refine the research question, conduct literature review and (if possible) gather initial datasets. \\ \hline

2,3* & \textit{First circle meeting, work plan} \\ \hline

3,4 & Develop the project plan, finalize methodology, start thinking of how to address text preprocessing (after having accessed and understood the data). Understand role of labelers/labeling process. Create GitHub repository for project. \\ \hline

5 & \textit{Informal meeting (Oct. 2nd), final project plan discussion.} \\ \hline \hline

5,6 & Preprocessing, store data in suitable format. Review Harry's work. Start looking into rule-based approaches. \\ \hline

7,8,9,10 & Implement and test rule-based methods (e.g. RegEx, NER, POS, etc.) for extracting key metadata. If time is left, start research on suitable ML approaches. \\ \hline

10,11,12* & \textit{Second circle meeting, midterm presentation} \\ \hline

11,12,13,14 & Develop and train machine learning models (e.g. Bert and others—research needed) and compare their performance to rule-based methods. \\ \hline

15,16 & Buffer weeks, allocate tasks depending on project progress. \\ \hline \hline

16,17* & \textit{Third circle meeting} \\ \hline

17,18 & Get final feedback, make adjustments, evaluate cross-project generalizability and conduct final evaluations. Write the thesis, incorporate feedback and prepare for final submission.\\ \hline

19 &  Thesis submission. \\ \hline

20,21* & \textit{Fourth circle meeting. Assessment meeting, final presentation (according to BEP Canvas page).} \\ \hline

22 & To be defined. \\ \hline

\end{tabular}
\caption{Project Timeline (*) overlapping activity: circle meetings}
\end{table}

\textbf{Research Questions}
\begin{itemize}
\item Primary RQ:
\textit{How can NLP techniques be applied to effectively* extract and structure key metadata from Dutch administrative decisions (energy permits), while overcoming the challenge of incomplete labels using weak labeling techniques, and comparing the performance of different approaches, including rule-based methods and machine learning?}

*\textit{\textbf{effectiveness}} refers to the ability of NLP techniques to extract and structure metadata from administrative decisions in a way that achieves satisfactory outcomes across multiple dimensions: accuracy, efficiency, scalability, and adaptability. An approach is considered effective if it yields high-quality results, is feasible to implement, and can be applied to various types of decisions with minimal supervision.

\item Sub-questions:
    \begin{itemize}
        \item \textit{How can weak labeling techniques be applied to systematically categorize Dutch administrative decisions, addressing the issue of incomplete data labeling (e.g., by leveraging legal keywords from document titles)?}
        Possible reflection points:
        \item \textit{Which NLP techniques (e.g., rule-based, machine learning, or others)—alone or in combination—are most effective in extracting key metadata from administrative decisions in the absence of fully labeled data?}
    \end{itemize}
    \item Points of relfection (while answering the questions):
    \begin{itemize}
        \item Specific challenges associated with processing legal language in Dutch?
        \item Generalization across different types of administrative decisions?
        \item For when dealing with ML approaches:  how can large language models (LLMs) be used for extracting key information from administrative decisions, and what strategies can be employed to ensure accuracy and reliability in their outputs?
        \item Non-quantitative evaluation: how? 
    \end{itemize}
\end{itemize}

\textbf{Proposed Methodology}
\setlength{\itemsep}{0pt} % Reduce space between list items
\begin{enumerate}
    \item Understand the project;
    \item Work on research question(s) (with further refining happening as the project advances);
    \item Store the data in a suitable format and understand its structure;
    \item Research available tools (preprocessing techniques, rule-based methods and ML libraries) and materials (related work, current developments in the field);
    \item Carry out experiments;
    \item Evaluate results (after choice of on appropriate evaluation system);
    \item Get feedback and incorporate it into the project;
    \item Report findings.
\end{enumerate}

\textbf{Literature Search}\\
So far, the literature review has focused on the papers recommended on the project page of the BEP Marketplace. As the project progresses, additional research will be conducted, particularly to address areas that prove more complex and require a deeper understanding of the relevant context and methodologies. I have also reviewed scientific and online articles on weak labeling, which have provided insights and resources into how to go about the lack labels in the dataset at hand.

\begin{itemize}
    \item Sansone, C. \& Sperlí, G. (2022). Legal information retrieval systems: State-of-the-art and open issues. \textit{Information Systems}, 106, 101967.
    \item Gray, M., Savelka, J., Oliver, W., \& Ashley, K. (2023). Can GPT alleviate the burden of annotation? In \textit{Legal Knowledge and Information Systems} (pp. 157–166). IOS Press.
    \item Zin, M. M., Nguyen, H. T., Satoh, K., Sugawara, S., \& Nishino, F. (2023). Information extraction from lengthy legal contracts: Leveraging query-based summarization and GPT-3.5. In \textit{Legal Knowledge and Information Systems} (pp. 177–186). IOS Press.
    \item Wolswinkel, C. J. (2024). Actieve openbaarmaking van beschikkingen. \textit{Nederlands Juristenblad}, volume 24 (pp. 1851–1857) (in Dutch only)\footnote{Read in machine translated version}.
    \item Lison, P., Barnes, J., \& Hubin, A. (2021). Weak Supervision Made Easy for NLP. 
    \item Ratner, A., Liss, S., Selsky, J., \& Snoek, J. (2017). Snorkel: Rapid Training Data Creation with Weak Supervision. In \textit{Proceedings of the 2017 ACM SIGKDD International Conference on Knowledge Discovery and Data Mining} (pp. 2201-2211). ACM.
\end{itemize}

\textbf{Website references (articles, blog posts, libraries) }

\begin{itemize}
    \item Humanloop. (2023). Why I changed my mind about weak labeling for ML. Retrieved from\\ \url{https://humanloop.com/blog/why-i-changed-my-mind-about-weak-labeling-for-ml}
    \item Weak labeling Python library (1): \texttt{Skweak}, \url{https://github.com/NorskRegnesentral/skweak}
    \item Weak labeling Python library (2): \texttt{Snorkel}, \url{https://www.snorkel.org/}
    \item \texttt{SpaCy} tutorial: \\
    \url{https://youtube.com/playlist list=PL2VXyKi-KpYvuOdPwXR-FZfmZ0hjoNSUo&si=MZ2zKrwSipQZL-E1}
\end{itemize}



 

























% ----------------end of document body---------------------

%---------------- start of references------------------

%\begin{thebibliography}{999}
%\bibitem[Faires \& Burden(1998)]{Faires}{Faires.D.J \& Burden.R. (1998). \textit{Numerical methods: second edition}. USA: Brooks/Cole Publishing Company}
%\bibitem[Griffiths \& Highams(2010)]{Griffiths}{Griffiths.D.F. \& Highams.D.J.(2010). \textit{Numerical methods for Ordinary Differential equations: Initial Value Problems}. London: Springer Undergraduate Mathematics series.}
% ---Not currently cited but kept in for future referance----\bibitem[Lambert(1973)]{Lambert1973}{Lambert.J.D.(1973). \textit{Computation Methods in Ordinary Differential Equations}, New York: John Wiley \& Sons.}
%\bibitem[Lambert(1991)]{Lambert1991}{Lambert.J.D.(1991). \textit{Numerical Methods for Ordinary Differential Systems: The Initial Value Problem}, West Sussex: John Wiley \& Sons Ltd.}


%\end{thebibliography}

%---------------- end of references------------------


\end{document}