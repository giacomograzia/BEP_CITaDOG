\documentclass[11pt]{article}

\usepackage[sort]{natbib}
\usepackage{fancyhdr}

% you may include other packages here (next line)
\usepackage{enumitem}



%----- you must not change this -----------------
\oddsidemargin 0.2cm
\topmargin -1.0cm
\textheight 24.0cm
\textwidth 15.25cm
\parindent=0pt
\parskip 1ex
\renewcommand{\baselinestretch}{1.1}
\pagestyle{fancy}
%----------------------------------------------------



% enter your details here----------------------------------

\lhead{\normalsize \textrm{BEP Draft Project Plan}}
\chead{}
\rhead{\normalsize TU/e SNR 1917811}
\lfoot{\normalsize \textrm{JBP000}}
\cfoot{}
\rfoot{Giacomo Grazia}
\setlength{\fboxrule}{4pt}\setlength{\fboxsep}{2ex}
\renewcommand{\headrulewidth}{0.4pt}
\renewcommand{\footrulewidth}{0.4pt}

	
\begin{document}


%----------------your title below -----------------------------

\begin{center}
{\bf BEP Draft Project Plan }
\end{center}


%---------------- start of document body------------------
\textbf{Project Planning}
\begin{table}[ht]
\centering
\begin{tabular}{|c|p{12cm}|}  % Adjust column width to fit the page
\hline
\textbf{Week} & \textbf{Task Description} \\ \hline
1,2 & Refine the research question, conduct literature review and (if possible) gather initial datasets. \\ \hline

2,3* & \textit{First circle meeting, work plan} \\ \hline

3,4 & Develop the project plan, finalize methodology, start thinking of how to address text preprocessing (after having accessed and understood the data). Understand role of labelers/labeling process.\\ \hline

5,6 & Preprocessing, store data in suitable format. \\ \hline

7,8,9,10 & Implement and test rule-based methods (e.g. RegEx, NER, POS, etc.) for extracting key metadata.\\ \hline

10,11,12* & \textit{Second circle meeting, midterm presentation} \\ \hline

11,12,13,14 & Develop and train machine learning models (e.g. Bert and others—research needed) and compare their performance to rule-based methods. \\ \hline

15,16 & Buffer weeks, allocate tasks depending on project progress. \\ \hline

16,17* & \textit{Third circle meeting} \\ \hline

17,18 & Get final feedback, make adjustments, evaluate cross-project generalizability and conduct final evaluations. Write the thesis, incorporate feedback and prepare for final submission.\\ \hline

19 &  Thesis submission. \\ \hline

20,21* & \textit{Fourth circle meeting. Assessment meeting, final presentation (according to BEP Canvas page).} \\ \hline

22 & To be defined. \\ \hline

\end{tabular}
\caption{Project Timeline (*) overlapping activity: circle meetings}
\end{table}

\textbf{Research Questions}
\begin{itemize}
\item Primary RQ:

“\textit{How can NLP techniques be effectively utilized to extract and structure key metadata from Dutch administrative decisions to improve their transparency and comparability?}”

\item (Possible) sub-questions:
    \begin{itemize}
        \item \textit{What are the specific challenges associated with processing legal language in Dutch?} \\ (e.g. any difficulty related to the language, availability of other studies/ML libraries to work with).
        \item \textit{Can the developed methods be generalized across different types of administrative decisions?} \\ (e.g. identification of recipient, reasons, topic, date, etc. of a non-legal document).
        \item \textit{How can rule-based methods and machine learning models be combined or compared to achieve the best results?} \\ (useful to compare different strategies, their advantages and disadvantages, their results).
        \item \textit{What are the most relevant NLP techniques for this task?} \\ (also related to previous sub-question).
        \item \textit{How can these techniques be tailored to specific types of administrative decisions?} \\(this is similar to the SQ on generalization of the methods to different types of administrative decisions, but allows to explain what are the specificities of the documents we are working with).
        \item \textit{What challenges arise in processing and interpreting legal or bureaucratic language?} \\ (can be incorporated to the SQ on the Dutch language and its challenges). 
    \end{itemize}
\end{itemize}


\textbf{Proposed Methodology}
\setlength{\itemsep}{0pt} % Reduce space between list items
\begin{enumerate}
    \item Understand the project;
    \item Work on research question(s) (with further refining happening as the project advances);
    \item Store the data in a suitable format and understand its structure;
    \item Research available tools (preprocessing techniques, rule-based methods and ML libraries) and materials (related work, current developments in the field);
    \item Carry out experiments;
    \item Evaluate results (after choice of on appropriate evaluation system);
    \item Get feedback and incorporate it into the project;
    \item Report findings.
\end{enumerate}

\textbf{Literature Search}\\
So far, the literature review has focused on the papers recommended on the project page of the BEP Marketplace. As the project progresses, additional research will be conducted, particularly to address areas that prove more complex and require a deeper understanding of the relevant context and methodologies. This ongoing literature exploration will help refine the approach and ensure that the project stays aligned with current developments in the field.

\begin{itemize}
    \item Sansone, C. \& Sperlí, G. (2022). Legal information retrieval systems: State-of-the-art and open issues. \textit{Information Systems}, 106, 101967.
    \item Gray, M., Savelka, J., Oliver, W., \& Ashley, K. (2023). Can GPT alleviate the burden of annotation? In \textit{Legal Knowledge and Information Systems} (pp. 157–166). IOS Press.
    \item Zin, M. M., Nguyen, H. T., Satoh, K., Sugawara, S., \& Nishino, F. (2023). Information extraction from lengthy legal contracts: Leveraging query-based summarization and GPT-3.5. In \textit{Legal Knowledge and Information Systems} (pp. 177–186). IOS Press.
    \item Wolswinkel, C. J. (2024). Actieve openbaarmaking van beschikkingen. \textit{Nederlands Juristenblad}, volume 24 (pp. 1851–1857) (in Dutch only)\footnote{Read in machine translated version}.
\end{itemize}



























% ----------------end of document body---------------------

%---------------- start of references------------------

%\begin{thebibliography}{999}
%\bibitem[Faires \& Burden(1998)]{Faires}{Faires.D.J \& Burden.R. (1998). \textit{Numerical methods: second edition}. USA: Brooks/Cole Publishing Company}
%\bibitem[Griffiths \& Highams(2010)]{Griffiths}{Griffiths.D.F. \& Highams.D.J.(2010). \textit{Numerical methods for Ordinary Differential equations: Initial Value Problems}. London: Springer Undergraduate Mathematics series.}
% ---Not currently cited but kept in for future referance----\bibitem[Lambert(1973)]{Lambert1973}{Lambert.J.D.(1973). \textit{Computation Methods in Ordinary Differential Equations}, New York: John Wiley \& Sons.}
%\bibitem[Lambert(1991)]{Lambert1991}{Lambert.J.D.(1991). \textit{Numerical Methods for Ordinary Differential Systems: The Initial Value Problem}, West Sussex: John Wiley \& Sons Ltd.}


%\end{thebibliography}

%---------------- end of references------------------


\end{document}